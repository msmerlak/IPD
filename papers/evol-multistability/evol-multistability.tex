\documentclass[12pt]{article}
\usepackage[utf8]{inputenc}
\usepackage{graphicx}
\usepackage{wrapfig}
\usepackage[margin = 3cm]{geometry}
\usepackage[safeinputenc, maxnames=10, backend = biber, sorting=none, url = false, doi = false, isbn = false, citestyle = numeric-comp]{biblatex}

\usepackage{changes}

\AtEveryBibitem{\clearlist{language}}
\AtEveryBibitem{\clearfield{note}}
\AtEveryBibitem{\clearfield{month}}
\AtEveryBibitem{\clearfield{publisher}}
\addbibresource{references.bib}
\renewbibmacro{in:}{}

\def\IPD{\textrm{IPD}}
\def\IPD1{\textrm{IPD}_1}

\title{Evolutionary multistability in the iterated prisoner's dilemma}

\author
{Matteo Smerlak\footnote{To whom correspondence should be addressed; E-mail:  smerlak@mis.mpg.de}\\
\\
\normalsize{Max Planck Institute for Mathematics in the Sciences,}\\
\normalsize{Inselstrasse 22, Leipzig, D-04103, Germany}\\
}

% Include the date command, but leave its argument blank.
\date{}


\begin{document}

\maketitle

\begin{abstract}
The iterated prisoner's dilemma (IPD) is a paradigmatic model of social dilemmas and the evolution of cooperation through reciprocity. Conventional wisdom has it that, while selfish (or even extortionary) strategies can give an edge in head-to-head competitions, cooperative behavior tends to be more evolutionarily stable in the long run. Here I show that the memory-one IPD---where phenotypes encode the stochastic response to just four possible joint plays---contains a phase where evolutionary dynamics is in fact multistable, with long periods of cooperation followed by rapid reversions to defection, and vice versa \emph{ad infinitum}. This surprising switch-like behavior in a strongly selected population shows that evolutionary dynamics does not always lead to evolutionarily stable states, even when phenotype space is small and easily explored. 
\end{abstract}

\section*{Introduction}

Evolution through natural selection is commonly viewed as a directed process  maximizing adaptation and reproductive success. Thus, in Wright's metaphor of the fitness landscape, evolving populations are pictured as climbing fitness peaks. While this view of evolution as optimization does not imply that all evolutionary change must be adaptive (some phenotypes appear to be byproducts of the evolution of other traits without being themselves adaptive), it does rely on the notion that directness and irreversibility are fundamental components of evolutionary change.  

Several well-known factors complicate this naive picture. For starters, genetic drift weakens the link between selective advantage and change in allele frequencies. In many (or most) cases, mutations with no fitness effect fix by chance, i.e. populations drift across neutral genotype networks instead of climbing fitness peaks; in small asexual populations, deleterious mutations can even displace fitter variants, a phenomenon known as Muller's ratchet. Second, adaptation is not always gradual or predictable: it is well known that many species have undergone long periods of evolutionary stasis followed by short bursts of rapid adaptation; as emphasized by Gould, the outcome (and not just the timing) of these bursts may be largely contingent: ``any replay of [life's] tape would lead evolution down a pathway radically different from the road actually taken''. The extent to which these contingencies are extrinsic (e.g. meteorites) or intrinsic (e.g. self-organized) is a largely open question. 

Third, and perhaps most importantly, the widespread occurrence of frequency dependence---the fact that reproductive success can depend on traits other than one's own---significantly muddles the picture of evolution as optimization, if only because a phenotype capable of resisting invasion by any possible mutant may not exist. As noted by Leimar, a ``Darwinian demon'' capable of controlling the mutational process could potentially direct the evolutionary process in any direction of its choosing, whatever the nature and direction of selective forces. In spite of these warnings, evolutionary game theory has mostly focused on defining, identifying and characterizing evolutionarily stable strategies, ideally in terms of static payoff comparisons. But is this focus on ESS as the endpoint of evolutionary trajectories always justified?   

To address some of these towering questions, it can be useful to focus on simple systems with clearly interpretable phenotypes and based on tractable mathematics. The memory-one iterated prisoner's dilemma ($\IPD1$) is one such system. In this game, two players X and Y cooperate (C) or defect (D) with a probability that depends on the outcome of their previous interaction. If X and Y cooperate, they both earn a reward ($R = 3$); if X cooperates and Y defects, then X is a `sucker' ($S = 0$) and Y has fallen to temptation ($T = 5$); and if X and Y both defect, they get the same punishment $P=1$. This simple rule gives rich to a surprisingly rich four-dimensional phenotype space $\Sigma_1 = [0, 1]^4$ (where the first component is to probability to cooperate after CC, the second component is the probability to cooperate after CD, etc.), comprising classical strategies such as Tit-For-Tat (TFT) and Win-Stay-Lose-Shift (WSLS) \cite{nowak_strategy_1993}. Remarkably, $\Sigma_1$ also contains extortionary strategies which take advantage of any opponent---no matter which strategy they choose---\cite{press_iterated_2012}, as well as generous strategies which guarantee that one's opponent never does worse than oneself \cite{stewart_extortion_2013}.

Identifying which kind of strategy is evolutionarily stable in $\IPD1$ is not just an interesting instance of the issue of directionality in evolution: it is also an important problem in itself whose solution may shed on the evolution of cooperation in biology and elsewhere. Based on the results of Axelrod's famous tournament, it was long believed that, being simple, nice and yet retaliatory, TFT epitomizes a stable strategy in the iterated prisoner's dilemma, and \emph{a fortiori} in $\IPD1$ \cite{axelrod_evolution_1981}; indeed, ecological simulations suggested it could rise to fixation in a field of diverse and sophisticated opponent strategies. The breakthrough discovery of so-called ``zero-determinant" extortionary strategies within $\Sigma_1$ by Press and Dyson \cite{press_iterated_2012} forced a reappraisal of this conclusion \cite{ball_physicists_2012}. Perhaps reassuringly, several studies found that more generous strategies can in fact replace extortionists in large populations \cite{adami_evolutionary_2013,stewart_extortion_2013}. Whether this scenario does indeed pan out, or defection takes root permanently, turns out depend on the mutation rate and the strength of selection \cite{stewart_extortion_2013, iliopoulos_critical_2010}; in particular, Iliopoulos \emph{et al.} identified a phase transition at a critical mutation rate $\mu_c$ between two phases, one dominated by cooperation ($\mu \leq \mu_c$, such that opponents are predictable enough to warrant the risk of cooperation), and the other by defection ($\mu \geq \mu_c$, in which opponents are too unpredictable).

Here I show that which strategy is more evolutionary stable in the $\IPD1$---which direction evolution chooses to go---is not always a well-posed question. I describe a phase of $\IPD1$, characterized by small mutational effects, weak genetic drift and no spatial structure, in which evolutionary dynamics is \emph{multistable} rather than convergent. This means that, instead of moving towards single attractor within $\Sigma_1$, evolutionary trajectories shift unpredictably between two long-lived but nevertheless transient states. The resulting dynamics resembles the burst noise observed in certain electronic devices, and can be modelled mathematically by a stationary, continuous-time Markov process over few effective states with exponentially distributed sojourn times. 

\section*{Results}

\paragraph*{The phase diagram of evolution in the IPD.} I performed evolutionary agent-based simulations similar to those described in \cite{iliopoulos_critical_2010}. Each agent embodies a stochastic strategy $p=(p_1, p_2, p_3, p_4)\in[0,1]^4 \equiv \Sigma_1$, where $p_1$ (resp. $p_2, p_3, p_4$) is the probability to cooperate with the opponent after the joint play CC (resp. CD, DC, DD). At each generation, each agent plays a match of the infinitely iterated prisoner's dilemma with all other members of a population of size $N$. After each match, the payoffs of a player playing strategy $p$ against strategy $q$ is increased by an amount $\pi(p,q)$ which can be computed using the Press-Dyson determinant formula \cite{press_iterated_2012}. Once all agents have played all their matches, strategies are mutated according to $p_i' = \min(\max(p_i + \sigma \epsilon, 0), 1)$, where $\sigma > 0$ represents a typical mutational effect and $\epsilon$ is a standard normal variable. After that, a new population of $N$ individuals is sampled with replacement, with a weight proportional to the payoff accrued in the previous generation (Wright-Fisher sampling). The entire process is repeated for a large number of generations. 

Iliopoulos \emph{et al.} describe the evolution of strategies in $\IPD1$ that are either cooperative or defective, depending on evolutionary parameters such as the mutation rate or the population replacement rate. I confirmed the existence of such a phase transition, measured here in terms of the rate of cooperation (the equilibrium probability of CC or CD plays) $\langle C\rangle_t$ averaged over players, opponents and evolutionary time as a function of $\sigma$ and $T$ (Fig. XXX). As noted in Ref. \cite{iliopoulos_critical_2010}, the transition from a cooperative phase under weak mutations to a defective phase under strong mutations has a natural interpretation: when future opponents can be reliably assumed to be similar to previous opponents, i.e. when the environment of players is sufficiently predictable, cooperation provides higher payoffs. But when mutational effects are large, there is no knowing how the next generation of opponents will play, and cooperation becomes unacceptably risky. In this phase evolution favors defection, at the cost of lower fitness for all.

Considering the variance of the mean population cooperation rate $\textrm{Var}_t(C)$ over time as a function of $\mu$ and $T$ reveals a richer picture. 


\paragraph*{Multistability and reduced effective dynamics.}

How can an evolutionary state be stable for long stretches of time, and yet eventually yield to a phenotypically very different invader population spawn by small mutations only? Standard tools of evolutionary game theory do not help here: one checks that all critical points of the adaptive dynamics $\dot{p} = \nabla_p \pi(p, q)_{\vert p = q}$, corresponding to a strong selection weak mutation regime, all lie on the boundary of $\Sigma_1$. 


\paragraph*{Strength of selection and the limits of adaptive dynamics.}



\paragraph*{Early-warning signals of evolutionary shifts.}

\section*{Discussion}

A large part of evolutionary theory in general, and evolutionary game theory in particular, has focused primarily on predicting the direction of evolution---or at least identifying the phenotypes that evolution favors, be them fitness peaks, evolutionarily stable strategies, or some other form of evolutionarily stable states. Viewed through this angle, the problem of the evolution of cooperation is to determine whether, or under which condition, cooperative behavior is an evolutionarily stable strategy. 

Previous work established that $\IPD1$ contains phases where evolutionary dynamics does not fall under this paradigm. For instance, Nowak and Sigmund show that, similar to the discrete-time logistic equation, a finite subset of $\Sigma_1$ consisting of $16$ special strategies, including TFT and WSLS, gives rise to chaotic dynamics under the replicator-mutator equation \cite{nowak_chaos_1993}, although this behavior likely does not extend to the full memory-one strategy space. Based on the results of agent-based simulations \cite{iliopoulos_critical_2010}, Adami \emph{et al.} point out another limitation of the concept of evolutionary stability in $\IPD1$: it does not make reference to the mutational process and fails to account for the role of mutational robustness \cite{adami_evolutionary_2016}. They show that, when mutation rates are sufficiently high, the most stable strategy is a "General Cooperator" which they characterize as a mutation rate-dependent ``evolutionarily stable quasistrategy" in reference to Eigen's notion of quasispecies \cite{eigen_molecular_1988}; this strategy outperforms in particular the generous strategies discussed by Stewart and Plotkin \cite{stewart_extortion_2013}. 

The results presented here show that Darwinian evolution can lead to yet another phase of evolutionary dynamics. In this phase, strategy space breaks up into a small number of (quasi)strategies with enhanced but imperfect stability. Over the long run of evolution, populations jump unpredictably between these states with timescales that depend on the mutation rate but not on the population size. This phase of evolution is not appropriately captured by the standard tools of evolutionary dynamics: neither the replicator-mutator equation nor the canonical equation of adaptive dynamics account for the bistability of cooperation and defection in the memory-one IPD. 

This switch-like behavior is consistent with the observation that rapid, major evolutionary change can occur without any exogenous driver. Various explanations for such kind of ``punctuated equilibrium" have been proposed. One, due to Sneppen and collaborators, relies on a simple but abstract and somewhat \emph{ad hoc} model of self-organized critical evolutionary dynamics in a one-dimensional fitness space \cite{sneppen_evolution_1995}. Another relates punctuations to the topology of genotype-to-phenotypes mappings: in very high-dimensional sequence spaces, it can take a long time for a population to diffuse along neutral networks until a new, fitter phenotype is found \cite{fontana_continuity_1998, crutchfield_evolutionary_2002,bakhtin_evolution_2021}. A third explanation, sometimes called ``quantum evolution'', depends on genetic drift to enable transient maladaptations before a new peak is found in the fitness landscape. In the multistable phase of $\IPD1$, rapid evolutionary change cannot be attributed to any of these causes. 

Overall, our findings reveal a new mode of evolutionary dynamics in the iterated prisoner's dilemma: one in which evolution has no attractor, no limit cycle and no preferred direction. In this phase, the evolutionary dynamics of cooperation may be compared with the reversible fluctuations of a physical system at thermal equilibrium---a process without history. 

\section*{Methods}

\paragraph*{Simulations.}
Agent-based simulations were coded and run using the Agents.jl julia package \cite{datseris_agents_2021}. 

\printbibliography


\end{document}
